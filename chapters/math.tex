% !TeX root = ../main.tex

\chapter{数学}

\section{数学符号和公式}

《撰写手册》要求数学符号要根据 GB/T 3102.11-1993《物理科学和技术中使用的数学符号》
\footnote{原 GB 3102.11-1993,根据2017年第7号公告和强制性标准整合精简结论,自2017年3月23日起,该标准转化为推荐性标准。}
使用,
这与 \LaTeX{} 默认的英美国家的数学符号习惯有所差异。
本模板基于 \pkg{unicode-math} 宏包配置数学符号,以遵循国标的规定:
\begin{enumerate}
  \item 大写希腊字母默认为斜体,如 \cs{Delta}:$\Delta$。
  \item 有限增量符号 $\increment$(U+2206)应使用 \cs{increment} 命令。
  \item 向量、矩阵和张量要求粗斜体,应使用 \cs{symbf} 命令,
    如 \verb|\symbf{A}|、\verb|\symbf{\alpha}|。
  \item 数学常数和特殊函数使用正体,
    如圆周率 $\symup{\pi}$、$\symup{\Gamma}$ 函数。
    应使用 \pkg{unicode-math} 宏包提供的 \cs{symup} 命令转为正体,
    如 \verb|\symup{\pi}|。
  \item 微分符号 $\mathrm{d}$ 使用正体,本模板提供了 \cs{dif} 命令。
\end{enumerate}

注意,\pkg{unicode-math} 宏包与 \pkg{amsfonts}, \pkg{amssymb}, \pkg{bm},
\pkg{mathrsfs}, \pkg{upgreek} 等宏包\emph{不}兼容。
本模板作了处理,用户可以直接使用 \cs{bm}, \cs{mathscr},
\cs{upGamma}。
关于数学符号更多的用法,参见 \pkg{unicode-math} 宏包的使用说明和符号列表
\pkg{unimath-symbols}。

在编辑数学公式时,最好避免直接使用字体命令,
而应该定义一些语义命令取代字体命令,
这样输入更简单,也让 \LaTeX{} 代码更有可读性,
而且还方便根据需要统一修改改格式。
参考示例文档中的 \file{math-commands.tex}

更多的例子:
\begin{equation}
  \upe^{\upi\uppi} + 1 = 0
\end{equation}
\begin{equation}
  \frac{\dif^2 u}{\dif t^2} = \int f(x) \dif x
\end{equation}
\begin{equation}
  \argmin_x f(x)
\end{equation}
\begin{equation}
  \mA \vx = \lambda \vx
\end{equation}



\section{量和单位}

宏包 \pkg{siunitx} 提供了更好的数字和单位支持:
\begin{itemize}
  \item \num{12345.67890}
  \item \num{1+-2i}
  \item \num{.3e45}
  \item \num{1.654 x 2.34 x 3.430}
  \item \si{kg.m.s^{-1}}
  \item \si{\micro\meter} $\si{\micro\meter}$
  \item \si{\ohm} $\si{\ohm}$
  \item \numlist{10;20}
  \item \numlist{10;20;30}
  \item \SIlist{0.13;0.67;0.80}{\milli\metre}
  \item \numrange{10}{20}
  \item \SIrange{10}{20}{\degreeCelsius}
\end{itemize}



\section{定理和证明}

示例文件中使用 \pkg{amsthm} 宏包配置了定理、引理和证明等环境。
用户也可以使用 \pkg{ntheorem} 宏包。

\newcommand\real{{\mathbf{R}}}

\begin{definition}
  If the integral of function $f$ is measurable and non-negative, we define
  its (extended) \textbf{Lebesgue integral} by
  \begin{equation}
    \int f = \sup_g \int g,
  \end{equation}
  where the supremum is taken over all measurable functions $g$ such that
  $0 \le g \le f$, and where $g$ is bounded and supported on a set of
  finite measure.
\end{definition}

\begin{assumption}
The communication graph is strongly connected.
\end{assumption}

\begin{example}
  Simple examples of functions on $\real^d$ that are integrable
  (or non-integrable) are given by
  \begin{equation}
    f_a(x) =
    \begin{cases}
      |x|^{-a} & \text{if } |x| \le 1, \\
      0        & \text{if } x > 1.
    \end{cases}
  \end{equation}
  \begin{equation}
    F_a(x) = \frac{1}{1 + |x|^a}, \qquad \text{all } x \in \real^d.
  \end{equation}
  Then $f_a$ is integrable exactly when $a < d$, while $F_a$ is integrable
  exactly when $a > d$.
\end{example}

\begin{lemma}[Fatou]
  Suppose $\{f_n\}$ is a sequence of measurable functions with $f_n \geq 0$.
  If $\lim_{n \to \infty} f_n(x) = f(x)$ for a.e. $x$, then
  \begin{equation}
    \int f \le \liminf_{n \to \infty} \int f_n.
  \end{equation}
\end{lemma}

\begin{remark}
  We do not exclude the cases $\int f = \infty$,
  or $\liminf_{n \to \infty} f_n = \infty$.
\end{remark}

\begin{corollary}
  Suppose $f$ is a non-negative measurable function, and $\{f_n\}$ a sequence
  of non-negative measurable functions with
  $f_n(x) \le f(x)$ and $f_n(x) \to f(x)$ for almost every $x$. Then
  \begin{equation}
    \lim_{n \to \infty} \int f_n = \int f.
  \end{equation}
\end{corollary}

\begin{proposition}
  Suppose $f$ is integrable on $\real^d$. Then for every $\epsilon > 0$:
  \begin{enumerate}
    \renewcommand{\theenumi}{\roman{enumi}}
    \item There exists a set of finite measure $B$ (a ball, for example) such
      that
      \begin{equation}
        \int_{B^c} |f| < \epsilon.
      \end{equation}
    \item There is a $\delta > 0$ such that
      \begin{equation}
        \int_E |f| < \epsilon \qquad \text{whenever } m(E) < \delta.
      \end{equation}
  \end{enumerate}
\end{proposition}

\begin{theorem}
  Suppose $\{f_n\}$ is a sequence of measurable functions such that
  $f_n(x) \to f(x)$ a.e. $x$, as $n$ tends to infinity.
  If $|f_n(x)| \le g(x)$, where $g$ is integrable, then
  \begin{equation}
    \int |f_n - f| \to 0 \qquad \text{as } n \to \infty,
  \end{equation}
  and consequently
  \begin{equation}
    \int f_n \to \int f \qquad \text{as } n \to \infty.
  \end{equation}
\end{theorem}

\begin{proof}
  Trivial.
\end{proof}

\newtheorem*{axiomofchoice}{Axiom of choice}
\begin{axiomofchoice}
  Suppose $E$ is a set and ${E_\alpha}$ is a collection of
  non-empty subsets of $E$. Then there is a function $\alpha
  \mapsto x_\alpha$ (a ``choice function'') such that
  \begin{equation}
    x_\alpha \in E_\alpha,\qquad \text{for all }\alpha.
  \end{equation}
\end{axiomofchoice}

\newtheorem{observation}{Observation}
\begin{observation}
  Suppose a partially ordered set $P$ has the property
  that every chain has an upper bound in $P$. Then the
  set $P$ contains at least one maximal element.
\end{observation}
\begin{proof}[A concise proof]
  Obvious.
\end{proof}
