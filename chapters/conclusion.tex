% !TeX root = ../main.tex

\chapter{总结}
\label{chap:conclusion}

为了使反向散射通信在非受控的Wi-Fi流量上更加可靠和高效,本工作提出了一个全新的对Wi-Fi流量进行预测的统计模型,同时在此模型的基础上对原有的反向散射通信协议进行了优化和扩展。对于Wi-Fi流量的建模使用了受马尔科夫链控制的信道模型,马尔可夫链的状态在有无之间以确定的概率1相互转换,不同的状态分别对应着持续时间的不同分布和不同的信道特征(普通高斯信道或“归零”信道)。基于该模型,本工作利用了蒙特卡罗方法在离散时间下对未来即将发送的数据的每一比特进行了有无状态概率预测。在此预测的基础上,LDPC的解码效果在最佳情况下得到了99.8\%的提升,同时利用置换分担错误的优化策略和根据当前环境中流量的变化进行参数的动态调节策略得以实现。通过使用在三种环境下采集的真实Wi-Fi流量数据,本工作对模型的预测和优化进行了半仿真测试。该模型在实验室环境下最高可实现平均98.6\%的预测击中率,新的LDPC编码最高将有效吞吐率提升了11kbps,动态码率调节机制使得有效吞吐量平均能达到最优值的80.79\%。总的来看,该模型可以实现比较精准的预测,并且可以对反向散射通信协议的在可靠性和有效性两方面进一步提高。在实际的反向散射标签上实现该预测和优化并进行相应评估将是下一步的目标。