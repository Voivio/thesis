% !TeX root = ../main.tex

\chapter{总结}
\label{chap:conclusion}

本工作提出了一个全新的对Wi-Fi流量进行预测的统计模型,同时在此模型的基础上对原有的反向散射通讯协议进行了优化和扩展。对于Wi-Fi流量的建模使用了受马尔科夫链控制的信道模型,马尔可夫链的状态在有无之间以确定的概率1相互转换,不同的状态分别对应着持续时间的不同分布和不同的信道特征(普通高斯信道或“归零信道”)。基于该模型,本工作利用了蒙特卡罗方法在离散时间下对未来即将发送的数据的每一比特进行了有无状态概率预测。在此预测的基础上,LDPC的解码得以实现,同时利用置换分担错误和根据当前环境中流量的变化进行码率的动态调节得以实现。通过三种环境下的测试,该模型在实验室环境下最高可实现平均98.6\%的击中率,新的LDPC编码最高将有效吞吐率提升了11kbps,动态码率调节在XX环境下将有效吞吐量提升了XX\%。该模型可以实现比较精准的预测,并且可以对反向散射通讯协议的表现进一步提高。在实际的反向散射标签上实现该预测和优化僵尸下一步的目标。