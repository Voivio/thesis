% !TeX root = ../main.tex

\chapter{相关工作}
\label{chap:related}

\textbf{环境反向散射}。
尽管以RFID为代表的反向散射通信在1948年便由Stockman等人提出\cite{stockman1948communication},利用环境中存在射频信号进行反向散射通信的环境反向散射通信在最近几年才得以发展。这种方法更加适用于低功耗设备,且无需布置特定的激励源设备。工作\cite{liu2013ambient}开创性地利用环境中的电视信号实现了反向散射标签之间的通信;
\textit{FM Backscatter}\cite{wang2017fm}%FM Backscatter
则实现了使用FM信号的反向散射通信。

但以上方法并不能提供反向散射设备与互联网的连通性,同时能接受响应信号的接收设备并不广泛。因此利用Wi-Fi进行反向散射通信逐渐得到了研究者的兴趣。工作\textit{BackFi}
\cite{bharadia2015backfi} %BackFi
设计了一种可以消除自干扰的机制,实现了在1米范围下5Mbps的高速率;
\textit{Wi-fi backscatter}\cite{kellogg2014wi}
通过散射改变当前的CSI/RSSI,从而使得Wi-Fi设备可以解码来自标签的信息;
\textit{Passive Wi-Fi}\cite{kellogg2016passive}%Passive Wi-Fi
利用反向散射产生可以直接被Wi-Fi设备解码的802.11b数据。更进一步,\textit{FreeRider}\cite{zhang2017freerider}
将蓝牙,802.11g/n WiFi以及ZigBee等信号拓展为可用的环境信号。但上述工作都对Wi-Fi信号源的传输进行了一定的控制,使用真实环境中的Wi-Fi信号进行反向散射通信在\cite{OFFB}中进行了探讨:Wi-Fi流量空白的阶段被视为突发错误,并利用RS码进行纠正。但在该工作中,没有对环境中Wi-Fi的流量进行进一步研究。本文对利用受马尔科夫链控制的信道模型对单纯使用里德-所罗门码的方案进行了进一步优化。并使得使用表现更好的LDPC码能够应用在反向散射通信中。

\textbf{Wi-Fi流量预测}。
本文的研究与在无线网络领域中流量预测的研究也有一定的关系。目前多数的研究,对Wi-Fi流量进行预测目的是\emph{避免Wi-Fi的干扰}。跨技术干扰(Cross technology interference)在2.4Ghz频段十分常见;Wi-Fi的广泛存在对许多同频段无线网络的表现产生影响。\textit{WISE}\cite{huang2010beyond}中,为了提升ZigBee的表现,作者建立了一个帕累托模型对Wi-Fi帧间间隔进行模拟,通过计算碰撞概率动态优化参数;工作\cite{dhanapala2017modeling}使用了受二阶马尔可夫模型控制的泊松过程
%(2nd order Markov Modulated Poisson Process , MMPP(2))
对Wi-Fi帧间间隔进行预测,从而实现干扰预测。然而,尽管这些工作对帧间间隔进行了模拟,却没有考虑Wi-Fi流量处于ON状态的统计情况,不能很好地应用于反向散射系统。

\textit{Glaze}\cite{kapetanovic2019glaze}与本工作比较相近。该工作希望完善反向散射系统中的下链路通信,同样使用马尔可夫链对当前信道中的Wi-Fi状态进行模拟。但在该工作中只是计算了短时段内Wi-Fi流量处于ON状态的占比,粒度较宽;本工作中的预测粒度更细,因此可以使得更多的优化如置换、LDPC解码效果提升得以实现。
