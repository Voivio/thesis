% !TeX root = ../main.tex
\chapter{背景介绍}
\label{chap:background}
\section{环境反向散射通信}
\begin{figure}
	\centering
	\includegraphics[width = .7\linewidth]{bkg_figure1_system-cropped}
	\caption{环境反向散射通信系统构成。标签可以对环境中的信号进行调制完成通信。}
	\label{fig:system}
\end{figure}
环境反向散射通信可以视为对以RFID为代表的传统反向散射通信的延伸。反向散射通信系统由三部分组成:激励源,反向散射标签和接收端。在传统的反向散射通信中,激励源是一个专用的设备,可以按照协议规定发送特定的电磁波信号;而在环境反向散射通信中,激励源是早已部署好的其他基础设施,如电视塔和广播塔等。反向散射标签从环境中已经存在的电磁信号中获取能量,并对其进行调制完成通信。图\ref{fig:system}展示了一个基于Wi-Fi信号进行反向散射通信的系统构成。Wi-Fi接入点在正常传输数据,反向散射标签对原始信号进行调制,将自身要传输的信息添加在原始信号中,特定的接收端对调制后信号进行接收,并完成信息的解码。

反向散射标签通过改变天线的阻抗实现对已有信号的调制。当电磁波遇到具有不同阻抗的介质交界面时,部分电磁波会被反射回来;而天线的阻抗与周围环境的阻抗不同,因此会有信号被天线反射。通过调整天线的阻抗可实现对反射回的电磁波能量的控制。当入射信号为$S_{in}$时,被反射回的信号$S_{out}$可以表示为:
\begin{equation}
S_{out} = S_{in} \frac{Z_a - Z_c}{Z_a + Z_c},
\end{equation}
其中$Z_a$是天线阻抗,$Z_c$是与天线相连的电路阻抗。通过将天线短接与否可以控制反射信号的能量大小:短接时反射全部信号,不短接时吸收信号。这两种状态的变化可以用来传递比特1和0。这种调制方案在反向散射通信中被广泛使用,称为开关键控(On-off keying, OOK)。假设背景流量的信号工作在频率$f_c$上,OOK本质上可看作使用一个频率为$\Delta f$的方波$S$与背景流量信号相乘。取方波的基波成分作为近似,调制后的信号可以描述为:
\begin{equation}
\begin{split}
	\sin(2\pi f_c t)\cdot S &\approx \sin(2\pi f_c t)\cdot \sin(2\pi \Delta_f t)\\
	& =\frac{\cos (2\pi (f_c - \Delta_f)t) - \cos (2\pi (f_c + \Delta_f)t)}{2}.
\end{split}
\end{equation}
因此,原始信号经过OOK调制后偏移到频率$f_c \pm \Delta_f$上,以减轻与原始信号之间的干扰。

\section{里德-所罗门码与突发错误}
\begin{figure}
	\centering
	\includegraphics[width = \linewidth]{bkg_figure2_RSburst-cropped}
	\caption{里德-所罗门码将突发错误转变成数量更少的码元错误,因此在恢复突发错误上具有优势。}
	\label{fig:rscode}
\end{figure}

里德-所罗门(RS)码是一种前向错误更正编码,在通信和媒体中被广泛地应用。从阿波罗登月计
划,到CD、DVD和蓝光光盘以及广播系统中DVB标准都有RS码的身影。 
一个里德-所罗门码$RS(n,k)$定义在有限域$\symup{F}$上,是一个总长度为$n$,信息长度为$k$,最短汉明距离为$n-k+1$的线性分组码;在实际应用中,有限域$\symup{F}$通常指定为$\symup{GF}(2^m)$,在这种情况下,每个码元都包含有$m$比特的信息。

$RS(n,k)$码非常适合纠正传输系统中的突发错误。以图\ref{fig:rscode}为例,假设当前环境中的Wi-Fi流量的有无(ON/OFF)状态如第二行所示,其中标x的部分表示当现比特遭遇了OFF状态。
如果我们使用$RS(7,3)$码进行编码,那么每3个比特构成一个码元,如第四行所示。尽管在比特上存在5个突发错误,但是在码元的角度来看只有2个码元发生了错误,因此$RS(7,3)$可以正确将该突发错误恢复。
不论一个码元中有多少个比特发生错误,在码元的角度看来都只发生了一个错误,因此RS码适合纠正传输系统中的突发错误,也因此适用于反向散射通信中Wi-Fi空闲状态造成的错误纠正。
\section{低密度奇偶校验码}

低密度奇偶校验码是一类在合适的构造情况下逼近香农极限的信道
编码。1962 年LDPC 码由 Gallager 提出,具有汉明距离良好、误
码率性能表现优异的特点。但是在当时受限于硬件实现上的困难,LDPC 码一直没有得到重视。后来,Tanner 从图的观
点对 LDPC 进行了全新的阐述;Mackay等人对LDPC码进行了重新的发现,自此该码收到更多的关注和发展。

%低密度奇偶校验码(LDPC)编码是一种接近信道容量的编码,在合适的构造情况下性能可以接近香农限。1963年,Gallager首次提出了LDPC码,但在当时由于实现上的困难一直没有得到重视。在1996年,Mackay等人对LDPC码进行了重新的发现,自此该码收到更多的关注和发展。

LDPC码基于具有稀疏矩阵性质的奇偶检验矩阵$H$建构而成。按照$H$各行1的个数(称为行重)和各列1的个数(称为列重)的特点,LDPC码可分为规则和非规则
两种。规则 LDPC 码中,行重相同且列重相同;如果行重或者列重不一致,该码则称为非规则LDPC码。在$H$中,
每一行代表一个对所有比特的校验,因此行重一致表示每个校验由相同个数的比特参与;而列重一致表明每个比特参与的校验个数是相同的。

与RS码不同的是,LDPC码的纠错能力并不能直接通过总长度和信息长度简单的计算出来。LDPC码与奇偶检验矩阵$H$本身的结构有关。因为
$H$的结构中包含了交织特性,在合适的构造情况下才能实现好的效果。LDPC码的构造有多种方法:包括结构化构
造方法和随机构造方法。前者在给定规则的条件下,对可行域进行搜索得到校验矩阵;后者在约束更少的条件下随机产生矩阵。尽管有一定的区别,但这两种方法都希望能在适当的复杂度下构造出具有较大
最小循环长度的$H$。其中最小循环长度是指$H$中所有的由1元素构成的环的最小长度。由于这方面的研究比较充分,在本文中$H$会选择已经构造好的
矩阵。但这会在码率的选择和信息长度、总长度的选择上受到限制。