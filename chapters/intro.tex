% !TeX root = ../main.tex

\chapter{简介}
\label{chap:introduction}

尼古拉·特斯拉曾在19世纪末畅想能量传输和通信能够摆脱线材的限制。我们现在所处的时代已经在无线通信和无线能量传输上取得了长足的进步——不论是广播,电视还是每天为人们服务的无线局域网;
近年来,物联网(Internet of Things)更是受到了学术界与工业界的关注。
物联网中,相邻物体可以共同协作和相互交流,且无需将接入电源或利用电池\cite{gershenfeld2004internet}
%N. Gershenfeld, R. Krikorian, and D. Cohen. The internet of things.Scientific American.
。如果这个愿景得以实现,更多的设备将融入到日常生活中,帮助人类在健康、仓储、智慧农业以及智慧城市等方面取得更进一步的发展。

但这些应用场景也对物联网设备的获能和通信能力提出了要求:物联网设备可能应用在各种环境之中,比如可穿戴设备\cite{baker2017internet}
% Baker, S. B., Xiang, W., & Atkinson, I. (2017). Internet of Things for smart healthcare: Technologies,challenges, and opportunities. IEEE Access, 5, 26521–26544
、仓储货物\cite{lee2018design}
% Design and application of Internet of things-based warehouse management system for smart logistics
等,这就对设备的尺寸、成本和重量等方面提出了限制。使用电池会使得设备在这些方面上不得不做出让步。从环境中获能啧可以让设备变得更轻、更易部署;另一方面,传统的射频通信需要设备主动产生无线电信号,这需要使用功耗较高的模拟电路元件,比如数模转换器,晶振和功率放大器等,仅从环境中获得的能量很难满足需求。这都对物联网设备的能量来源和通信方式都提出了新的挑战。

环境反向散射技术(Ambient backscattering)可以很好的获能和通信上能量限制的问题。使用此技术的设备可以从环境中存在的射频信号获取能量,并对这些信号进行简单的调制完成通信。由于无需自主产生信号,比起传统的无线通信方式(如Wi-Fi和LTE)在能量上节约了数个量级;同时相较传统的反向散射通信(如RFID),也免除了搭建特定功能的基础设施的成本\cite{liu2013ambient}
%Ambient back
。近些年来,研究者们对利用电视信号\cite{liu2013ambient},广播信号\cite{wang2017fm}
%A. Parks and J. Smith. Sifting through the airwaves: Efficient and scalable multiband rf harvesting. In IEEE RFID 2014,
或者基站信号\cite{parks2014sifting}进行环境反向散射通信做出了许多研究。但广为部署的Wi-Fi愈来愈受研究者们的青睐。目前,关于Wi-Fi反向散射的工作中大多要求信号源是“受控的”——即对其进行了特殊的设置,让这些设备按照预想的状态发送数据。现实中,Wi-Fi接入点在大多数情况下是非受控的,由于带有冲突避免的载波监听多路访问(CSMA/CA)协议和当前数据交换的繁忙程度不同,常常会出现没有Wi-Fi数据包的空闲(OFF)状态存在。这便在以上工作在现实中应用时可能出现性能上的下降。

%在工作\cite{}中达到了1米距离下5Mbps的高传输速度。然而,该工作中Wi-Fi信号源需要运行特定的程序传输受控的流量序列;

在本文中,我们试图解决如何在Wi-Fi信号源非受控的情况下,更可靠更高效地进行反向散射通信。工作\cite{OFFB}跨出了从实验室到实际使用的一步。该工作将实际情况下Wi-Fi空闲状态看为正常传输中出现的突发错误,并设计了一个节省能量的里德-所罗门码(Reed-Solomon code)对丢失的信息进行恢复。受到该工作启发,我们提出了一个受马尔科夫链控制的模型对信道进行建模,该模型统计环境中Wi-Fi流量的历史信息,用来计算未来传输中会遇到突发错误的概率。预测得到的概率可以对传输过程中的参数进行动态调整(如帧长、速率和码率等),从而提升了系统的可靠性和高效性。同时,借助此模型,我们可以使用更多种类的编码来提升系统的可靠性,比如低密度奇偶校验(LDPC)编码。

%建立了一个受马尔可夫过程控制的加性高斯信道模型,从而实现了对Wi-Fi背景流量的预测,从而实现了。

总结来说,本文的贡献包括:

\begin{itemize}
	\item 对Wi-Fi流量进行了统计建模,建立了一个受马尔可夫过程控制的信道模型。可以实现对未来环境中的Wi-Fi流量进行预测和模拟;
	
	\item 在该模型的基础上,本工作进一步优化使用里德-所罗门码进行通信的方案(后简称为RS方案),实现了更高的可靠性(可以处理更多的“突发错误”),以及更高的效率(动态调节传输参数);
	
	\item 在该模型的基础上,本工作进一步拓展了RS方案,具有更高纠错能力的低密度奇偶校验码可以在反向散射通讯中使用。
	
	\item 本工作对以上内容进行了半仿真模拟完成评估,即在真实采集的Wi-Fi流量上进行仿真模拟。结果表明该模型和基于该模型的优化可以实现可靠性和效率上的提升。
\end{itemize}

此外,本工作提出的模型不仅可以对通信系统进行优化,也可以用来对从Wi-Fi信号中获能的系统进行优化。但这不在本文的讨论范围之内。

本文按照以下内容展开:第\ref{chap:background}章介绍了环境反向散射技术和相关编码的背景内容;第\ref{chap:model}章介绍了对非受控情况下的Wi-Fi流量的统计建模;第\ref{chap:optimization}章介绍了基于此模型实现的流量预测以及对参数的动态优化机制;第\ref{chap:evaluation}章介绍了对使用该模型实现的预测和优化效果的评估结果。相关工作的介绍与本文内容的总结位于第\ref{chap:related}章和第\ref{chap:conclusion}章。
