% !TeX root = ../main.tex

\chapter{简介}
\label{chap:introduction}

尼古拉·特斯拉曾在19世纪末畅想能量传输和通信能够摆脱线材的限制。我们现在所处的时代已经在无线通信和无线能量传输上取得了长足的进步——不论是广播,电视还是每天为服务的无线局域网;
而近年来,物联网(Internet of Things)更是受到了学术界与工业界的关注:
相邻的物体之间可以共同协作和相互交流,而且无需将这些设备接入电源或维持电池\cite{}。如果这个愿景得以实现,更多的设备将会融入到我们的日常生活中,帮助我们在健康、仓储、智慧农业以及智慧城市等方面取得更进一步的发展\cite{}。

而这些应用场景也对物联网设备的获能和通信能力提出了要求:物联网设备可能出现在各种环境之中,包括可穿戴设备\cite{}、仓储货物\cite{}等,这就对设备的尺寸、成本和重量这些方面都有了限制。电池的使用会影响这些方面,从环境中获能更受青睐;而另一方面,传统的射频通信需要设备主动产生无线电信号,而这种方法需要功耗较高的模拟电路元件包括数模转换器(DACs),晶振以及功率放大器等\cite{}。尺寸、成本和重量上的限制对物联网设备的能量来源和通信方式都提出新的挑战。

环境反向散射技术(Ambient backscattering)可以很好的通信上能量限制的问题。它主要利用对已存在的射频信号获取能量并对这些信号进行反向散射完成通信,从而比起传统的无线通信方式如Wi-Fi和LTE在能量上节约数个量级;同时相较传统的反向散射通信如RFID,也免除了搭建特定的功能基础设施的成本\cite{}。近些年来,研究者们对利用电视信号\cite{},广播信号\cite{}或者基站信号\cite{}进行环境反向散射通信做出了许多研究。但广为布置的Wi-Fi愈来愈受研究者们的青睐。但目前,关于Wi-Fi反向散射的工作中大多要求信号源是“受控的”——即我们可以对其进行了特殊的设置,让这些设备按照预想的状态发送数据。现实中,Wi-Fi接入点在更多的情况下是不受控的,由于带有冲突避免的载波监听多路访问(CSMA/CA)协议和当前数据交换的繁忙程度不同,有时会有没有Wi-Fi数据包的空闲状态存在。这便在实验室中的高速率和现实中的应用之间留下了障碍。

%在工作\cite{}中达到了1米距离下5Mbps的高传输速度。然而,该工作中Wi-Fi信号源需要运行特定的程序传输受控的流量序列;

在本文中,我们试图解决如何在Wi-Fi信号源非受控的情况下,更快更可靠地实现反向散射通信。工作\cite{}跨出了从实验室到实际使用的一步。该工作将实际情况下Wi-Fi空闲状态看为正常传输中出现的突发错误,并设计了一个节省能量的里德-所罗门码(Reed-Solomon codes)对丢失的信息进行恢复。受到该工作启发,我们借助对环境中Wi-Fi流量的历史统计信息对即将发送的数据帧可能会遇到的突发错误进行预测,并基于这个预测对当前传输过程中帧长、速率和码率的动态调整,进一步提高了系统的吞吐量。同时,借助对信道的建模,我们可以使用更多种类的编码来提升系统的可靠性,比如LDPC编码。

%建立了一个受马尔可夫过程控制的加性高斯信道模型,从而实现了对Wi-Fi背景流量的预测,从而实现了。

总结来说,本文的贡献包括:

\begin{itemize}
	\item 对Wi-Fi流量进行了统计建模,建立了一个受马尔可夫过程控制的信道模型。可以实现对未来环境中的Wi-Fi流量进行预测和模拟;
	
	\item 在该模型的基础上,本工作进一步优化使用里德-所罗门码进行通信的方案(后简称为RS方案),实现了更高的可靠性(可以处理更多的“突发错误”),以及更高的效率(动态调节帧长和码率);
	
	\item 在该模型的基础上,本工作进一步拓展了RS方案,具有更高纠错能力的低密度奇偶校验码(LDPC)可以在当前设定下使用。
\end{itemize}

此外,本文进行额建模不仅可以对通信系统进行优化,也可以用来对从Wi-Fi信号中获能的系统进行优化。但这不在本文的讨论范围之内。

本文按照一下内容展开:第\ref{chap:background}章介绍了环境反向散射技术的背景和相关的编码;第\ref{chap:model}章介绍了对非受控情况下的Wi-Fi流量的统计建模;第\ref{chap:optimization}章介绍了基于此模型实现的流量预测机制以及对参数的动态优化;第\ref{chap:evaluation}章介绍了对该模型的预测和优化效果的评估结果。相关工作的介绍与总结位于第\ref{chap:related}章和第\ref{chap:conclusion}章。
